\section{Satellite galaxies around a massive central - part 2}

In this section we look at question 1 of the third hand-in. 
I have mostly copy-pasted my functions from the tutorials, where I worked together with my sister,
Evelyn van der Kamp (s2138085), so some of our functions are quite similar.
I have also copied functions from the previous hand-ins, such as Romberg integration and LU decomposition.

\lstinputlisting{NURHW3LizQ1.py}

\subsection{Question 1a}

In this section I took my function from the previous hand-in for $4 \pi n(x) x^2$, now with A given.
I also made a function which returns $-4 \pi n(x) x^2$, because finding the minimum of that function is the same as finding the maximum of the positive function.
I imported my golden section search algorithm and called it to search for the minimum between 0 and $x_{max} = 5$, with as my first guess for the maximum $x=1$.
I took my target accuracy to be $10^{-20}$ and the maximum iterations to be 100.
I found the following maximum and function value at the maximum:

\lstinputlisting{Maximumoutput.txt}

Here, the first value shows the radius at the maximum, and the second value is $N(x)$ at the maximum.


\subsection{Question 1b}

For this question we aim to calculate the best fit parameters using a $\chi^2$ method. 
First, I import the functions needed for the Levenberg-Marquardt routine, which is the function for LU decomposition (using the improved Crout's algorithm), functions to swap, scale and add rows, a function calculating the $\chi^2$ value, and finally the functions for the LM routine, including two functions to calculate the $\alpha$ matrix and $\beta$ from the algorithm.
Other than that, I also import my Romberg integration function to calculate the normalization factor $A$ as well as the variance $\sigma_i^2$.

I created a new function to fit as well as the derivatives with respect to the model parameters $a$, $b$, and $c$.
I plotted the histograms of the files without predetermined bins and I saw that the maximum radius these satellites reach is about $x = 2.4$, so I took my bin range to be [0,2.4] in real space, since the histogram already followed a nice Poisson distribution in real space.
I took 24 bins because that looks like the bins show the least variation.

The mean number of satellites is the total number of satellites divided by the number of halos, and the mean number of satellites per halo in bins is given by the total number of satellites in a certain bin divided by the total number of satellites and the binwidth.
I took [2,0.5,1.6] as my initial guess for $a$, $b$, and $c$, because it is close to the given values from 1(a), where I took into account that my bins only go to $x = 2.4$.

I calculated an initial normalization factor $A$ for these parameters as well as initial variances, by integrating over the whole range [0,5] and integrating over the bins respectively.
Then I called my LM function to calculate new solutions, and incorporated new $A$ and variance calculations inside this function for each new solution. 
After a maximum of 50 iterations, I do a final calculation of $A$ and then calculate the predicted values in each bin, but since my calculation of $A$ does not give a precise enough normalization to make the sums of the observed values and the predicted values match up, I scale my predicted bin values so that the total sum matches the observed amount.

The plots showing the best fit results can be seen in Figure \ref{fig:LMfit}

\begin{figure}[ht!]
\begin{center}
        \begin{subfigure}{.49\textwidth}
      \centering
      \includegraphics[width=\linewidth]{LMplot11.png}
    \end{subfigure}
    %
    \vspace{-10pt}
    %
    \begin{subfigure}{.49\textwidth}
      \centering
      \includegraphics[width=\linewidth]{LMplot12.png} 
    \end{subfigure}
    %
     \vspace{-10pt}
    %
    \begin{subfigure}{.49\textwidth}
      \centering
      \includegraphics[width=\linewidth]{LMplot13.png} 
    \end{subfigure}
    %
     \vspace{-10pt}
    %
    \begin{subfigure}{.49\textwidth}
      \centering
      \includegraphics[width=\linewidth]{LMplot14.png} 
    \end{subfigure}
    %
     \vspace{-10pt}
    %
    \begin{subfigure}{.49\textwidth}
      \centering
      \includegraphics[width=\linewidth]{LMplot15.png} 
    \end{subfigure}
    %
    \caption{Log-log plots showing the five datasets and the $\chi^2$ best fit with Levenberg-Marquardt routine. The legend shows $<N_{sat}>$ as well as the best fit values and the obtained $\chi^2$ value.}
    \label{fig:LMfit}
\end{center}
\end{figure}

The precise results will be shown in the next section, together with the Poisson fit results.

\subsection{Question 1c}
For this section, we want to calculate the best fit parameters again, but instead of minimizing a $\chi^2$, we want to maximize a Poisson likelihood.
We can do this by minimizing minus the natural log of the Poisson likelihood.

For this part, I imported a function which applies a Quasi Newton method to minimize in 3D. 
This function uses the Hessian, as well as the gradient/the derivatives of the log likelihood function with respect to the model parameters to find the minimum. 
It also uses golden section search to perform line minimization.
I created the log likelihood function as well as a function for the gradient of that function, used the same initial guess for $a$, $b$, and $c$, and applied the Quasi Newton method to find the best fit.
Here I calculated $A$ inside the likelihood function instead of inside the minimizing function.
Afterwards, I did the same as in the previous section: calculated $A$ a final time, calculated the predicted bin values and scaled those to match the observed sum of bin values, and plotted the results.

\begin{figure}[ht!]
\begin{center}
        \begin{subfigure}{.49\textwidth}
      \centering
      \includegraphics[width=\linewidth]{LMplot11.png}
    \end{subfigure}
    %
    \vspace{-10pt}
    %
    \begin{subfigure}{.49\textwidth}
      \centering
      \includegraphics[width=\linewidth]{LMplot12.png} 
    \end{subfigure}
    %
     \vspace{-10pt}
    %
    \begin{subfigure}{.49\textwidth}
      \centering
      \includegraphics[width=\linewidth]{LMplot13.png} 
    \end{subfigure}
    %
     \vspace{-10pt}
    %
    \begin{subfigure}{.49\textwidth}
      \centering
      \includegraphics[width=\linewidth]{LMplot14.png} 
    \end{subfigure}
    %
     \vspace{-10pt}
    %
    \begin{subfigure}{.49\textwidth}
      \centering
      \includegraphics[width=\linewidth]{LMplot15.png} 
    \end{subfigure}
    %
    \caption{Log-log plots showing the five datasets and the Poisson likelihood best fit with a Quasi Newton routine. The legend shows $<N_{sat}>$ as well as the best fit values and the obtained -ln$L$ value.}
    \label{fig:QNfit}
\end{center}
\end{figure}

The plot generated can bee seen in Figure \ref{fig:QNfit}.
The results of the fitting are given by:

Dataset 1:
\lstinputlisting{Fitresultsoutput11.txt}
Dataset 2:
\lstinputlisting{Fitresultsoutput12.txt}
Dataset 3:
\lstinputlisting{Fitresultsoutput13.txt}
Dataset 4:
\lstinputlisting{Fitresultsoutput14.txt}
Dataset 5:
\lstinputlisting{Fitresultsoutput15.txt}

Here, the first value shows $<N_{sat}>$, the second to fourth values show $a$, $b$, and $c$ obtained by the $\chi^2$ fit, the fifth value shows the minimum $\chi^2$ value, the sixth to eighth values show $a$, $b$, and $c$ obtained by the Poisson -ln$L$ fit, and the last value shows the minimized \ln$L$ obtained.


\subsection{Question 1d}

In this section, we were asked to perform a G-test and calculate the Q values corresponding to the $\chi^2$ fit. 
I took the number of satellites in a bin instead of the mean number, to make sure that we have integers for $O_i$, and I scaled my fitted values to match (multiplied by the total number of halos and binwidth).
Since there were some zero values for the number of observed values in a bin, so I made a for-loop to calculate the sum, and added zero when $O_i = 0$.
Here, the number of independent degrees of freedom $k$ is given by the number of bins, minus 3 because we want to calculate 3 fit parameters ($a$, $b$, and $c$), and then minus 1 because we want the independent degrees of freedom, and the last bin is uniquely determined by the previous bins once we know the fit parameters.

Dataset 1:
\lstinputlisting{Resultsoutput11.txt}
Dataset 2:
\lstinputlisting{Resultsoutput12.txt}
Dataset 3:
\lstinputlisting{Resultsoutput13.txt}
Dataset 4:
\lstinputlisting{Resultsoutput14.txt}
Dataset 5:
\lstinputlisting{Resultsoutput15.txt}

Here, the first values are the G test results, for the $\chi^2$ and Poisson fits respectively, and the second values are the Q values for both of the fits.
It seems like the fits are basically equivalent between Poisson and $\chi^2$, which is a bit strange, since the $\chi^2$ fit should give a biased result. 
All of them give Q values of very close to one, but the G values are pretty high for the 11-13 (first three) files, while 14 & 15 (last two) fits give better G values and also are a better fit to the data.

